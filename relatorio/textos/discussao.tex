\section{Análise dos Resultados}

    \begin{figure}[H]
        \centering
        \caption{Tempo de execução em relação a quantidade de clientes e o número de variáveis de otimização.}

        \includepgf{tempo_variaveis.pgf}
    \end{figure}

    Infelizemente, não conseguimos executar todos os experimentos. Um dos possíveis problemas é que o número de variáveis no modelo, que é dado por:
    \[
        \text{número de variáveis} = P L F + P F J = P F (L + J)
    \]

    Considerando os limites da instância no enunciado, temos que
    \begin{align*}
        5 \cdot J (5 + J) \leq &\text{número de variáveis} \leq 10 \cdot 2 J (10 + J) \\
        5 J^2 + 25 J \leq &\text{número de variáveis} \leq 20 J^2 + 200 J
    \end{align*}

    Portanto, o número de variáveis cresce de forma quadrática com o parâmetro $J$. Considerando que, no melhor dos casos, o tempo de execução depende linearmente do número de variáveis, então o tempo também cresce de forma quadrática.
