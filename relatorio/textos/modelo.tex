\section{Modelo}

    \subsection{Descrição do Problema}

        Uma companhia possui $F$ fábricas para atender a demanda de $J$ clientes. Cada fábrica pode escolher dentre $L$ máquinas e $M$ tipos de matéria-prima para produzir $P$ tipos de produtos. A companhia precisa desenvolver um plano de produção e transporte com o objetivo de minimizar os custos totais. Mais especificamente, a companhia deve determinar a quantidade de cada tipo de produto a ser produzida em cada máquina de cada fábrica e a quantidade que deve ser transportada de cada produto partindo de cada fábrica para cada consumidor.

    \subsection{Parâmetros}

        INTEIROS, DIM

        \begin{description}
            \item[$J$] quantidade de clientes;
            \item[$F$] quantidade de fábricas;
            \item[$L$] quantidade de máquinas em cada fábrica;
            \item[$M$] quantidade de tipos de matéria-prima;
            \item[$P$] quantidade de tipos de produtos;
        \end{description}

        REAIS

        \begin{description}
            \item[$D_{j,p}$]    demanda do cliente j, em toneladas, do produto $p$;
            \item[$r_{m,p,l}$]  quantidade de matéria-prima $m$, em toneladas, necessária para produzir uma tonelada do produto $p$ na máquina $l$;
            \item[$R_{m,f}$]    quantidade de matéria-prima $m$, em toneladas, disponível na fábrica $f$;
            \item[$C_{l,f}$]    capacidade disponível de produção, em toneladas, da máquina $l$ na fábrica $f$;
            \item[$p_{p,l,f}$]  custo de produção por tonelada do produto $p$ utilizando a máquina $l$ na fábrica $f$;
            \item[$t_{p,f,j}$]  custo de transporte por tonelada do produto $p$ partindo da fábrica $f$ até o cliente $j$;
        \end{description}

    \subsection{Variáveis de Decisão}

        \begin{description}
            \item[$x_{p,l,f}$] toneladas produzidas de $p$ na máquina $l$ da fábrica $f$;
            \item[$y_{p,f,j}$] toneladas transportadas de $p$ da fábrica $f$ para o cliente $j$;
        \end{description}

        \subsubsection{}

    \subsection{Restrições}

        \begin{description}
            \item[Não-negatividade] \begin{align*}
                x_{p,l,f} \geq 0
                && &\text{para toda produção de $p$ na máquina $l$ da fábrica $f$} \\
                y_{p,f,j} \geq 0
                && &\text{para toda transporte de $p$ da fábrica $f$ para o cliente $j$}
            \end{align*}
            \item[Atendimento às demandas dos clientes] \begin{align*}
                \sum_{f = 1}^F y_{p,f,j} = D_{j,p}
                && \text{para todo cliente $j$ e produto $p$}
            \end{align*}
            \item[Limite de matéria-prima disponível] \begin{align*}
                \sum_{p = 1}^P \sum_{l = 1}^L r_{m,p,l} x_{p,l,f} \leq R_{m,f}
                && \text{para toda matéria-prima $m$ e fábrica $f$}
            \end{align*}
            \item[Capacidade de produção] \begin{align*}
                \sum_{p = 1}^P x_{p,l,f} \leq C_{l,f}
                && \text{para toda máquina $l$ na fábrica $f$}
            \end{align*}
            \item[Equivalência de produção e transporte] \begin{align*}
                \sum_{l = 1}^L x_{p,l,f} = \sum_{j = 1}^J y_{p,f,j}
                && \text{para toda produção de $p$ na fábrica $f$}
            \end{align*}
        \end{description}

    \subsection{Função Objetivo}

        \begin{align*}
            \text{custo de produção} &= \sum_{f = 1}^F \sum_{p = 1}^P \sum_{l = 1}^L  x_{p,l,f} p_{p,l,f} \\
            \text{custo de transporte} &= \sum_{f = 1}^F \sum_{p = 1}^P \sum_{j = 1}^J y_{p,f,j} t_{p,f,j}
        \end{align*}

        \begin{align*}
            \text{custo total} &= \text{custo de produção} + \text{custo de transporte} \\
            &= \sum_{f = 1}^F \sum_{p = 1}^P \left(\sum_{l = 1}^L  x_{p,l,f} p_{p,l,f} + \sum_{j = 1}^J y_{p,f,j} t_{p,f,j}\right)
        \end{align*}
