\section{Introdução} \label{sec:introducao}

    Uma companhia possui $F$ fábricas para atender a demanda de $J$ clientes. Cada fábrica pode escolher dentre $L$ máquinas e $M$ tipos de matéria-prima para produzir $P$ tipos de produtos. A companhia precisa desenvolver um plano de produção e transporte com o objetivo de minimizar os custos totais. Mais especificamente, a companhia deve determinar a quantidade de cada tipo de produto a ser produzida em cada máquina de cada fábrica e a quantidade que deve ser transportada de cada produto partindo de cada fábrica para cada consumidor. Os parâmetros do problema encontram-se abaixo:

    \begin{description}
        \item[$D_{j,p}$]   demanda do cliente j, em toneladas, do produto $p$;
        \item[$r_{m,p,l}$] quantidade de matéria-prima $m$, em toneladas, necessária para produzir uma tonelada do produto $p$ na máquina $l$;
        \item[$R_{m,f}$]   quantidade de matéria-prima $m$, em toneladas, disponível na fábrica $f$;
        \item[$C_{l,f}$]   capacidade disponível de produção, em toneladas, da máquina $l$ na fábrica $f$;
        \item[$p_{p,l,f}$] custo de produção por tonelada do produto $p$ utilizando a máquina $l$ na fábrica $f$;
        \item[$t_{p,f,j}$] custo de transporte por tonelada do produto $p$ partindo da fábrica $f$ até o cliente $j$;
    \end{description}
