\documentclass[assignment = 1]{homework}

\usepackage{caption, subcaption, pdfpages, float}
\usepackage{graphics, wrapfig, pgf, graphicx}
\usepackage{enumitem}
\graphicspath{{../resultados/}}


% pacotes para importar código
\usepackage{caption, booktabs}
\usepackage[section, newfloat, outputdir=out]{minted}
\definecolor{sepia}{RGB}{252,246,226}
\setminted{
    bgcolor = sepia,
    style   = pastie,
    frame   = leftline,
    autogobble,
    samepage,
    python3,
    breaklines
}
\setmintedinline{
    bgcolor={}
}

% ambientes de códigos de Python
\newmintedfile[pyinclude]{python3}{}
\newmintinline[pyline]{python3}{}
\newcommand{\pyref}[2]{\href{#1}{\texttt{#2}}}

% \SetupFloatingEnvironment{listing}{name=Código}
% \captionsetup[listing]{position=below,skip=-1pt}

\usepackage{csquotes}
\usepackage[style=verbose-ibid,autocite=footnote,notetype=foot+end,backend=biber]{biblatex}
\addbibresource{referencias.bib}
\usepackage[section]{placeins}

\usepackage[hidelinks]{hyperref}
\usepackage[noabbrev, nameinlink, brazilian]{cleveref}
\hypersetup{
    pdftitle  = {MO824 - Atividade 1},
    pdfauthor = {187679 e 206734}
}

\newcommand{\textref}[2]{
    \hyperref[#2]{#1 \ref*{#2}}
}

\renewcommand{\vec}[1]{\mathbf{#1}}

\DeclareMathOperator{\round}{round}

\usepackage{import, multirow}
\usepackage{tikz}
\usetikzlibrary{matrix}
\usetikzlibrary{positioning}
\usetikzlibrary{automata}
\usetikzlibrary{shapes}

\newenvironment{kmatrix}[1][1.3cm]{
    \begin{tikzpicture}[node distance=0cm]
        \tikzset{square matrix/.style={
                matrix of nodes,
                column sep=-\pgflinewidth, row sep=-\pgflinewidth,
                nodes={draw,
                    minimum height=#1,
                    anchor=center,
                    text width=#1,
                    align=center,
                    inner sep=0pt
                },
            },
            square matrix/.default=#1
        }
}{
    \end{tikzpicture}%
}

\newcommand*{\Scale}[2][4]{\scalebox{#1}{\ensuremath{#2}}}%

\newcommand{\red}[1]{\textcolor{red}{\textbf{#1}}}
\def\qm{?}


\begin{document}

    \pagestyle{main}

    \section{Modelo}

    \subsection{Descrição do Problema}

        Uma companhia possui $F$ fábricas para atender a demanda de $J$ clientes. Cada fábrica pode escolher dentre $L$ máquinas e $M$ tipos de matéria-prima para produzir $P$ tipos de produtos. A companhia precisa desenvolver um plano de produção e transporte com o objetivo de minimizar os custos totais. Mais especificamente, a companhia deve determinar a quantidade de cada tipo de produto a ser produzida em cada máquina de cada fábrica e a quantidade que deve ser transportada de cada produto partindo de cada fábrica para cada consumidor.

    \subsection{Parâmetros}

        INTEIROS, DIM

        \begin{description}
            \item[$J$] quantidade de clientes;
            \item[$F$] quantidade de fábricas;
            \item[$L$] quantidade de máquinas em cada fábrica;
            \item[$M$] quantidade de tipos de matéria-prima;
            \item[$P$] quantidade de tipos de produtos;
        \end{description}

        REAIS

        \begin{description}
            \item[$D_{j,p}$]    demanda do cliente j, em toneladas, do produto $p$;
            \item[$r_{m,p,l}$]  quantidade de matéria-prima $m$, em toneladas, necessária para produzir uma tonelada do produto $p$ na máquina $l$;
            \item[$R_{m,f}$]    quantidade de matéria-prima $m$, em toneladas, disponível na fábrica $f$;
            \item[$C_{l,f}$]    capacidade disponível de produção, em toneladas, da máquina $l$ na fábrica $f$;
            \item[$p_{p,l,f}$]  custo de produção por tonelada do produto $p$ utilizando a máquina $l$ na fábrica $f$;
            \item[$t_{p,f,j}$]  custo de transporte por tonelada do produto $p$ partindo da fábrica $f$ até o cliente $j$;
        \end{description}

    \subsection{Variáveis de Decisão}

        \begin{description}
            \item[$x_{p,l,f}$] toneladas produzidas de $p$ na máquina $l$ da fábrica $f$;
            \item[$y_{p,f,j}$] toneladas transportadas de $p$ da fábrica $f$ para o cliente $j$;
        \end{description}

        \subsubsection{}

    \subsection{Restrições}

        \begin{description}
            \item[Não-negatividade] \begin{align*}
                x_{p,l,f} \geq 0
                && &\text{para toda produção de $p$ na máquina $l$ da fábrica $f$} \\
                y_{p,f,j} \geq 0
                && &\text{para toda transporte de $p$ da fábrica $f$ para o cliente $j$}
            \end{align*}
            \item[Atendimento às demandas dos clientes] \begin{align*}
                \sum_{f = 1}^F y_{p,f,j} = D_{j,p}
                && \text{para todo cliente $j$ e produto $p$}
            \end{align*}
            \item[Limite de matéria-prima disponível] \begin{align*}
                \sum_{p = 1}^P \sum_{l = 1}^L r_{m,p,l} x_{p,l,f} \leq R_{m,f}
                && \text{para toda matéria-prima $m$ e fábrica $f$}
            \end{align*}
            \item[Capacidade de produção] \begin{align*}
                \sum_{p = 1}^P x_{p,l,f} \leq C_{l,f}
                && \text{para toda máquina $l$ na fábrica $f$}
            \end{align*}
            \item[Equivalência de produção e transporte] \begin{align*}
                \sum_{l = 1}^L x_{p,l,f} = \sum_{j = 1}^J y_{p,f,j}
                && \text{para toda produção de $p$ na fábrica $f$}
            \end{align*}
        \end{description}

    \subsection{Função Objetivo}

        \begin{align*}
            \text{custo de produção} &= \sum_{f = 1}^F \sum_{p = 1}^P \sum_{l = 1}^L  x_{p,l,f} p_{p,l,f} \\
            \text{custo de transporte} &= \sum_{f = 1}^F \sum_{p = 1}^P \sum_{j = 1}^J y_{p,f,j} t_{p,f,j}
        \end{align*}

        \begin{align*}
            \text{custo total} &= \text{custo de produção} + \text{custo de transporte} \\
            &= \sum_{f = 1}^F \sum_{p = 1}^P \left(\sum_{l = 1}^L  x_{p,l,f} p_{p,l,f} + \sum_{j = 1}^J y_{p,f,j} t_{p,f,j}\right)
        \end{align*}

    \section{Experimentos}

    A partir da escolha de $J$, a instância é gerada com valores aleatórios seguindo as restrições do enunciado. Tudo foi feito em Python 3, inclusive a implementação do modelo.

    \subsection{Detalhes da Máquina}

        \begin{minted}{bash}
            $ uname -a
            Linux marmis-arch 5.17.1-zen1-1-zen #1 ZEN SMP PREEMPT Mon, 28 Mar 2022 21:56:46 +0000 x86_64 GNU/Linux
            $ lscpu | grep -E 'Architecture|Model name|CPU max MHz'
            Architecture:                    x86_64
            Model name:                      AMD Ryzen 5 1500X Quad-Core Processor
            CPU max MHz:                     3500,0000
        \end{minted}

    \subsection{Resultados}

        \begin{table}[H]
            \centering
            \begin{tabular}{ccccc|cc|cc|c}
                \toprule\toprule
                $J$ & $F$ & $L$ & $M$ & $P$ & Vars. & Restrs. & Iters. & Tempo de Execução & Custo da Solução \\
                \midrule
                % 100 & 116 & 6 & 6 & 6 & 350 & 2688 & 1110 & 13.51 s & 187 573.8 \\
                % 200 & 294 & 7 & 8 & 9 & 813 & 8856 & 7228 & 13.5 min & 553 319.9 \\
                100 & 126 & 5 & 8 & 5 & 367 & 2768 & 12753 & 8.9 s & 160 268.5 \\
                200 & 299 & 5 & 8 & 5 & 813 & 6382 & 58689 & 5 min 13 s & 309 782.7 \\
                300 & \multicolumn{4}{c|}{---} & \multicolumn{2}{c|}{---} & \multicolumn{2}{c|}{---} & --- \\
                \bottomrule\bottomrule
            \end{tabular}
        \end{table}


\end{document}
